% Ejemplo de documento LaTeX
% Tipo de documento y tamaño de letra
\documentclass[12pt]{article}
% Preparando para documento en Español.
% Para documento en Inglés no hay que hacer esto.
\usepackage[spanish]{babel}
\selectlanguage{spanish}
\usepackage[utf8]{inputenc}
% EL titulo, autor y fecha del documento
\title{Manual breve de los comandos de Bash}
\author{ María Inés Valenzuela C.}
\date{30 de enero del 2015}
% Aqui comienza el cuerpo del documento
\begin{document}
% Construye el título
\maketitle
\section{¿Qué es {\tt bash}?}
BASH es un shell de Unix (intérprete de comandos de Unix) escrito para el proyecto GNU, es el shell por defecto en la mayoría de sistemas Linux.Su función es de mediar entre el usuario y el sistema.
\section{Comandos Básicos}
\subsection{Navegación}
\begin{tabular}{|c|l|l|}
\hline
Comando & Descripción & Ejemplo \\
\hline
pwd  & Muestra la ruta del directorio actual &  pwd \\ \hline
ls & Enlista los archivos dentro de un directorio & ls Descargas \\ \hline
ls -ld & Te permite ver los permisos de un archivo especifico & \\ \hline
ls -a & Enlista los archivos incluyendo los alchivos ocultos & \\ \hline
cd & Te cambia de un directorio a otro & cd Notas \\
\hline
echo & Se utiliza para desplegar mensajes & echo SHELL \\ \hline
less & Visualiza página a página un archivo & less Notas.txt \\ \hline
file & Muestra el tipo de un archivo & file Notas.txt  \\ \hline
\end{tabular}
\subsection{Manipulación de archivos}
\begin{tabular}{|c|l|l|}
\hline
Comando & Descripción & Ejemplo \\
\hline
cp & Copia un archivo o directorio & cp Notas Descargas \\ \hline
mkdir &Crea un nuevo directorio & mkdir Notas \\ \hline
rmdir & Borra un directorio & rmdir Notas \\ \hline
mv & Mueve un archivo & mv filename1 filename2 \\ \hline
touch & Crea un archivo en blanco & touch mina.txt \\ \hline
rm & Elimuna un archivo & rm mina.txt \\ \hline
vi & Permite editar archivos & vi notas.txt \\ \hline
cat & Te permite ver un archivo. & cat notas.txt\\ \hline
head & Muestra el inicio de un archivo & head prog1.c \\ \hline
\end{tabular}
\subsection{Ayuda con comandos}
\begin{tabular}{|c|l|l|}
\hline
Comando & Descripción & Ejemplo \\ \hline
man & Te da una descripción de cada comando & man mkdir \\ \hline
type & Muestra información sobre el tipo de comando & type mkdir \\ \hline
which & localiza un comando & which ls \\ \hline
\end{tabular}
\subsection{Permisos}
\begin{tabular}{|c|l|l|}
\hline
Comando & Descripción & Ejemplo \\ \hline
chmod & Cambia los permisos de un archivo & chmod +x miscript \\ \hline
chown & Cambia el dueño un archivo & chown nobody miscript\\ \hline
\end{tabular}
\subsection{Permisos}
\begin{tabular}{|c|l|l|}
\hline
Comando & Descripción & Ejemplo \\ \hline
.  & Es referencia al directorio actual & ./INES \\ \hline
.. & Te regresa a una jerarquia arriba & ../Desktop \\ \hline
\end{tabular}
% Nunca debe faltar esta última linea.
\end{document}
